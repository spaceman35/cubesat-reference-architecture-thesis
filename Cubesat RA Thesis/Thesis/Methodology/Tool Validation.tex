(Is Methodology present tense? past tense?)
Before the Reference Architecture is ready for teams to begin using, a full test was conducted to validate the tool and ensure everything worked as planned. Grissom-P is a mission that students were assigned in a previous sequence, and it is also a real world AFIT mission with real requirements and documentation, so it was a perfect test bed for this Reference Architecture. It is also a unique mission, given that it has two distinct and physically separated payloads, so it tested the modularity of the Reference Architecture. Near the end of the Reference Architecture design process, Grissom P was used as an example to run through the Reference Architecture quickly to ensure each step was working properly. This highlighted several issues that needed to be fixed before the faculty demonstration, and then, once the Reference Architecture was ready, Grissom-P was fully fleshed out using the Reference Architecture. Furthermore, this was done by another graduate student, which helped prove that someone unfamiliar with the architecture could use and understand it. 

The ultimate test will be when a new cohort of students use the tool. Lessons learned from these system designs will improve the model going forward to address any remaining gaps or adapt to changing requirements. This is the whole point of a Reference Architecture, after all.