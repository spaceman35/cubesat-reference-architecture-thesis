Modeling styles vary from person to person and organization to organization, so external feedback was desired for this Reference Architecture to ensure it made sense to others. To accomplish this, the model was first demonstrated to other students who previously took AFIT's Space Vehicle Design sequence, and they were asked to model a system using the tool. This peer feedback process led to many clarifications and tweaks, and their models were the impetus for many of the provided value properties. Furthermore, their common questions were addressed in the included help guide. Technicians who work on the AFIT CubeSat program were also helpful. Understanding what they look for and what they call components and subsystems motivated some design changes to remain as consistent as possible.

After getting peer feedback, the model was demonstrated to faculty members who will teach the courses in the Space Vehicle Design sequence. Of the three instructors, only one has significant modeling experience, so this model and included guidance needed to be usable by students without requiring faculty help for normal modeling questions. The primary inputs required from the faculty were the inputs to the Document Generators. Because the faculty members decide the format and objectives for each deliverable report, they were given a chance to provide comments or changes to the relevant documents that this Reference Architecture will generate for their classes. If these requirements change in the future, which is highly likely, the students have been provided guidance for how to make a new template or modify an existing template so the instructors will not have to understand the underlying template code.

Finally, the CubeSat Reference Architecture is being used by the current cohort of students in the course sequence. When the first course started, they were given a lengthy recorded demonstration of the model, with guidance for how to use the cloud environment, how to use the document generators, and how to use and tailor the template model for their unique missions. During the duration of the course, they have an avenue to ask questions and receive help with the model, which may also lead to changes or improvements in the core Reference Architecture. 