%This is the scope assumption 
This research was primarily intended to aid student design teams in a University setting, and AFIT's space vehicle design series of courses is a perfect test-bed for this. AFIT's first space vehicle design course teaches and implements MBSE for stakeholder analysis and requirements generation; however, the following courses do not continue the use of the model for the actual design and implementation of the CubeSat. The goal of this research is to create a useful Reference Architecture to aid students in designing the physical satellite and tracing system requirements down to the component level. This Reference Architecture should be useable even by users not so familiar with MBSE, and it should assist with the system-level review process including Critical Design Reviews, Test Readiness Reviews, etc. Even though AFIT students will be the first users of this Reference Architecture, it is generic enough to be used for any team wishing to develop a CubeSat program from the ground up. It has all the functionality needed to develop requirements, design the physical system, and perform basic simulations. It also features helpful resources like a component library to assist with the physical design and document generators to create tailored stakeholder documents from model elements. 

A Reference Architecture offers a baseline template for students to build from, using lessons learned from past projects and creating the framework to streamline the design process. A large effort of this research was focused on creating a generic model with default component specifications throughout to spark ideas in the brainstorming process for students and aid in system analysis. Another component of this research was creating basic analysis capabilities within the model, allowing students to tweak component specifications to see how those changes affect overall capabilities and requirements. Additionally, the model traces the analysis to template requirements that future students will tailor for their unique projects. This allows for rapid simulations of key performance parameters or measures of effectiveness for the system. 

Furthermore, additional work is being done using this Reference Architecture for more in-depth state analysis and integration with \abbreviationFull[Systems Tool-Kit]{STK} and MATLAB, so it's critical to form a robust baseline to build off of.

In order to test the validity of the tool, examples of this Reference Architecture will first be demonstrated to the relevant course instructors to show how it could be used by students. Feedback will be incorporated into the model before being used by future classes. Additionally, a comprehensive how-to guide and modeling style guide will be provided to students to walk through the process using a generic design. 
