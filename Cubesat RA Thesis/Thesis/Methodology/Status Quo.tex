As discussed in Chapter \ref{Intro}, the Reference Architecture is intended to improve CubeSat system designs by providing a starting point and a framework that guides teams through the entire systems engineering process. The tool will have AFIT students in mind with some course-specific features, but will also work as a general CubeSat Reference Architecture outside of AFIT. To understand the organization and unique features of this Reference Architecture, it's helpful to understand the primary goals, inputs, and outputs of the course sequence. 

At AFIT, the first course starts with teams given a \abbreviationFull[Mission Capabilities Document]{MCD} for a fictional mission, which outlines the "Mission Need Statement," "Operational Context," and a set of required capabilities and design constraints. From this set of inputs, students develop stakeholder concerns and needs, perform trade studies, write a \abbreviationFull[Concept of Operations]{CONOPS}, and finally develop a set of mission requirements. These artifacts are carried into the next course, where system level requirements are defined and a physical structure is designed. Finally, in the third course, students take those system-level requirements and further define subsystem-level requirements and develop test plans to verify those requirements. These courses are intended to flow together, and this Reference Architecture will help by providing the model framework to carry between and support all three courses. Throughout the course sequence, there are also milestone reviews and stakeholder documentation requirements to fulfill. 

Students in this course sequence use the textbook Space Mission Engineering: The New Space Mission Analysis and Design (cite) by Wertz, et al. Wertz focuses mainly on the requirements definition and validation portion of the Systems Engineering process, so other more general Systems Engineering texts were consulted to supplement Wertz, such as those by Friedenthal, Buede, Maier, and Rechtin. 

\textcolor{red}{add the table of Space Mission Engineering Processes from Wertz?}

The Space Vehicle design sequence, as taught by AFIT, has one primary input, the MCD. From there, the following outputs are generated as part of the process. Each report, trade study, and review will have a place in the Reference Architecture.

\textcolor{red}{Should I make this a table instead with a column to describe each one?}

\textbf{Reports:}
\begin{itemize}
    \item Mission Capabilities Document
    \item Stakeholder Analysis Report
    \item Mission Requirements Document
    \item Concept of Operations
    \item Space Vehicle Requirements Document
    \item Operational Requirements Document
    \item Subsystem Test Plans
    \item Subsystem Test Report
    \item Flight Readiness Review Report
\end{itemize}

\textbf{Trade Studies:}
\begin{itemize}
    \item Constellation Trade Study
    \item Launch Vehicle Trade Study
    \item RF Link Budget Analysis
    \item Mass Budget
    \item Power Budget
    \item Cost Budget
\end{itemize}

\textbf{Reviews:}
\begin{itemize}
    \item Mission Concept Review
    \item Preliminary Design Review
    \item Critical Design Review
    \item Test Readiness Review
    \item Flight Readiness Review
\end{itemize}

This list is if course not exhaustive, as differing missions or stakeholders may want different outputs, so the Reference Architecture is designed to be flexible enough for unforeseen changes. In addition to these formal documents and reviews, the model itself is useful for describing the physical decomposition and interfaces of the system. The model itself holds all the text, figures, tables, and trade studies that are used in the documents as well. For example, the CONOPS document goes through mission and fault phases, describing subsystem conditions, detailing activity diagrams for those phases, and writing narratives to describe activities. These are all contained in model elements, and the document just calls these elements in the appropriate format for display. 

Table \ref{table:CubeSat Development Process} briefly outlines what the typical CubeSat development process looks like, and \ref{table:AFIT CubeSat Development Process} shows the process done in the short time frame of the design sequence at AFIT. 

\begin{table}[h!]
\centering
\begin{tabular}{|l|l|c|} 
 \hline
 Step & Project Phase & Typical Timeframe \\ [0.5ex] 
 \hline\hline
 1 & Concept Development & 1-6 months\\
 2 & Securing Funding & 1-12 months\\
 3 & Merit and Feasibility Review & 1-2 months\\
 4 & CubeSat Design & 1-6 months\\
 5 & Development and Submittal of Proposal & 3-4 months\\
 6 & Selection and Manifesting & 1-36 months\\
 7 & Mission Coordination & 9-18 months\\
 8 & Licensing & 4-5 months\\ 
 9 & Flight Specific Documentation Development & 10-12 months\\
 10 & Ground Station Design, Development and Test & 2-12 months\\ 
 11 & CubeSat Hardware Fabrication and Testing & 2-12 months\\
 12 & Mission Readiness Review & Half day\\
 13 & CubeSat to Dispenser Integration and Testing & 1 day\\
 14 & Dispenser and Launch Vehicle Integration & 1 day\\
 15 & Launch & 1 day\\
 16 & Mission Operations & Variable, up to 2 years\\
 \hline
\end{tabular}
\caption{Typical CubeSat Development Process}
\label{table:CubeSat Development Process}
\end{table}


\begin{table}[h!]
\centering
\begin{tabular}{|l|l|c|} 
 \hline
 Step & Project Phase & Typical Timeframe \\ [0.5ex] 
 \hline\hline
 fill & this in & later\\
 \hline
\end{tabular}
\caption{AFIT CubeSat Development Process}
\label{table:AFIT CubeSat Development Process}
\end{table}