As launch service providers continue to offer more ride-sharing opportunities, access to space has never been more available or affordable for small satellites. The nanosatellite size (1-10 kg) (cite) has exploded in popularity over the last two decades, and among that size class, the CubeSat construct has become the de facto standard. A CubeSat is a sizing standard defined in 1999 by California Polytechnic State University and Stanford University's \abbreviationFull[Space and Systems Development Laboratory]{SSDL}, with a basic "1U" unit being 10 cm x 10 cm x 10 cm and a mass less than 1.33 kg \citep{DesignSpec}. An example 1U CubeSat is shown in Figure \ref{fig:1U CubeSat Example}. Furthermore, CubeSats are defined by how many 1U cubes they contain. For example, a 3U CubeSat would be three 1U cubes together, and a 6U CubeSat is six 1U cubes combined, as shown in Figure \ref{fig:6U CubeSat Example}. This standardized sizing framework allows for rapid prototyping with common chassis and common dispenser mechanisms, and this drives down the cost of research and development for these CubeSats. CubeSats routinely use \abbreviationFull[Commercial Off The Shelf]{COTS} components to further drive down development costs. California Polytechnic Institute also publishes CubeSat Design Specifications for 1U-3U CubeSats and for 6U CubeSats to assist design teams \citep{DesignSpec}.

\begin{figure}[!h]
    \centering
    \includegraphics[width=3in]{Thesis/Literature_Review/Lit Review Figures/vanderbiltcubesat.jpg}
    \caption{1U CubeSat Example}
    \label{fig:1U CubeSat Example}
\end{figure}

\begin{figure}[!h]
    \centering
    \includegraphics[width=3in]{Thesis/Literature_Review/Lit Review Figures/6Ucubesatbus.jpg}
    \caption{6U CubeSat Example}
    \label{fig:6U CubeSat Example}
\end{figure}

A primary benefit of the CubeSat standard is the lower cost of both the satellite hardware and of the launch costs. The cost of failure for a CubeSat is orders of magnitude less than for a large, exquisite satellite, so CubeSats offer a proving ground for maturing technologies. A traditional satellite requires a dedicated launch vehicle, a distinct payload adapter, and millions to billions of dollars in research and development. By contrast, a CubeSat might only cost \$100,000 to \$500,000 in research and development costs, and the launch cost can be less than \$1 Million (cite). Even more valuable than the reduced cost is the ability to flight test articles in the space environment to iterate and mature technologies. Many materials, sensors, and other components have been matured through CubeSats. For example, the Air Force Academy's FalconSat-7 was designed to get flight heritage on a polyimide photon sieve and determine its imaging performance before being used in future operational satellites \citep{FalconSat7}. Their previous mission, FalconSAT-6, was designed to improve \abbreviationFull[Hall Effect Thruster]{HET} technologies and low power communication options \citep{FalconSat6}. 

Furthermore, as resiliency in space becomes more important, CubeSats offer a solution that is attracting research for military application. As CubeSats are so small, a mission could include many individual CubeSats as a system, or "swarm," to create a large constellation that drastically increases the overall reliability and resiliency for the mission. In the private sector, a notable example is the Swarm SpaceBee, a 0.25U CubeSat that is part of a 150-CubeSat constellation in \abbreviationFull[Low Earth Orbit]{LEO}, testing out global \abbreviationFull[Internet of Things]{IOT} tracking of ships, vehicles, and other remote sensors \citep{Harris2019}. 

Finally, Launch Service Providers are routinely offering ride-share opportunities as secondary customers, with some launches even accommodating more than 60 payloads. SpaceX launched SSO-A in 2018 which carried 15 microsatellites (10-100 kg) and 49 CubeSats, which came from universities and other research institutes from around the world including the previously mentioned FalconSat-6 \citep{eoPortal}. This CubeSat standard and the increasing demand for small satellites in orbit has lowered the barrier to entry, allowing universities and small research teams to develop their own space programs. In fact, AFIT has its own CubeSats in development, including the "Grissom" 6U bus, which will form the foundation for several distinct CubeSat variations. 

Due to the unique advantages that CubeSats offer for both the Department of Defense and to small university teams, AFIT has embraced the concept and is preparing graduate students for future jobs in satellite acquisitions using CubeSats as the primary tool. Developing a CubeSat is a daunting task, especially for students without satellite experience, so the MBSE method is first taught to students before applying it to CubeSat design.





