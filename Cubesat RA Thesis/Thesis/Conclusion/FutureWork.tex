% Future Work

One of the primary goals for this CubeSat Reference Architecture was to establish the platform for future work. Some of that work has already begun, including an Integrated Mission Modeling Tool that uses the physical structure in the Reference Architecture to create detailed MATLAB Simulink and STK simulations for mission modeling. These tools will improve the fidelity of mission simulations and provide visual views of the orbits for ground contacts, while also simulating multiple payloads at once.

The Reference Architecture is meant to be improved and adapted over time. As new teams use the model, they will be creating new physical blocks for components they chose, and they will be creating new constraint blocks for analysis. These can be saved in the component library for future reuse, so over time, the component library can grow and contain more "plug and play" blocks. Eventually, the component library should have a variety of components for each subsystem to choose from, and there should be analysis blocks to tailor depending on the mission's requirements. 

There are also some gaps in the Reference Architecture that can be tackled by other researchers in the future. For example, this current iteration focuses on verifying subsystem level requirements with hardware tests, but most mission level or system requirements are not properly accounted for. This was due to the specific requirements of the Spacecraft Design Sequence at AFIT, but additional functionality can be built in to verify requirements at the mission or system level for teams who have a need for that information. Furthermore, only minimal risk functionality has been provided. Currently, a user can assign a risk level to a requirement, but there is no place to describe that risk or risk mitigation steps. 
