\begin{abstract}

The CubeSat class of nanosatellites has lowered the barrier of entry to space and has rapidly gained popularity in recent years. The lower development cost, small form factor, and reuse of commercial off-the-shelf components makes the CubeSat form factor an ideal platform for University teams, where budget and development time are extremely limited. To successfully design a CubeSat system in a rapid cycle conducive to academic timelines, a Reference Architecture geared towards University CubeSat development would be helpful. A Reference Architecture would speed up the development process by providing a template, capturing previous work and lessons learned from subject matter experts, providing a framework to focus on the CubeSat’s design rather than the fine details of modeling software. A Reference Architecture can also add functionality that student teams could use and improve over time, such as pre-built analysis functions and a library of components to choose from. This thesis presents a CubeSat Reference Architecture designed to meet these needs and explores its unique features, diagrams, and custom libraries. The CubeSat Reference Architecture was validated by relevant course instructors and is being used by a cohort of students in the Spacecraft Design Sequence at AFIT.


\end{abstract}