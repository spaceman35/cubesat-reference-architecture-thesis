As discussed in Chapter \ref{LitReview}, a similar effort has already been taking place with Small Unmanned Aerial Systems at AFIT. Those efforts created a Reference Architecture for a similar design course sequence, so the first step was to explore that Reference Architecture and get some ideas and any lessons learned from that effort. Of primary note was their component library, which allows the SUAS designer to choose from commonly available components to rapidly prototype a new system. Their organization was also well done, with top-level pages to show internal structures and a package breakdown to separate Requirements, Structure, Behavior, and Analysis. Several of these organization practices will be expanded upon in this CubeSat Reference Architecture.

While not explicitly stated as such, students going through this course sequence learned the OOSEM approach to modeling systems, so that approach should be used for this Reference Architecture as well. To bridge the gap between Wertz' Firefly model \citep{Wertz2011SpaceSMAD} and the OOSEM methodology, Friedenthal's text \citep{FriedenthalArchitectingSpacecraft} was used as a reference, as he also uses OOSEM in his approach. Looking at these models provided a good foundation upon which to start building a Reference Architecture. Wertz had detailed subsystem breakdowns and relevant calculations, Friedenthal explained the OOSEM process and how it relates to CubeSat designs, and the SUAS model \citep{Jacques2019} helped guide the organizational structure and capabilities of a Reference Architecture.

As this tool is meant to encourage new designs and not stifle creativity, there are some architectural design considerations when building the Reference Architecture. How detailed should it be? Should internal block diagrams be filled out? Should sate machines and mission phase descriptions come fully described? These considerations are key points of discussion with the faculty that will teach these courses, and these points will be discussed later. Additionally, the Reference Architecture project used teamwork and input from faculty, lab technicians, and other students who previously went through this program.

Once completed, this Reference Architecture would be used from the very beginning of the design course sequence all the way through its conclusion. They will be given the Reference Architecture file with their mission-specific MCD and some guidance, and then they design the system from the ground up using that template. 

