Designing a spacecraft is a daunting and complex endeavor. Due to the nature of space launch, most spacecraft only get one chance at success, and spacecraft can take many years and millions of dollars to develop. As such, modeling, simulation, and testing are vital for a space vehicle program's success, and finding new ways to mature technologies and flight test them can improve this process. The CubeSat-class of nanosatellite can help by providing a cost-effective platform to mature technologies or even perform operational missions as part of a CubeSat constellation. This thesis attempts to assist design teams in rapidly developing and prototyping these CubeSat designs. 

Dr. Will Roper, the former assistant secretary of the Air Force for acquisition, has emphasized the need for a faster acquisition cycle and for bolder ideas. During the Air Force Association’s Air, Space and Cyber Conference in 2019, Dr. Roper said “To become a more competitive acquisition system, the Air Force needs to be aware of trends in technology. The world is changing. We have to change with it. The key is to decide which technology will be successful and being able to act on those trends with a system that is leaner, meaner and faster than our opponents.” \citep{Roper2019} In the space domain, CubeSats are that latest technological "leaner and meaner" trend, and the US Air Force and Space Force are embracing it. Additionally, CubeSats are becoming increasingly popular  in the commercial sector around the world, with the number of CubeSat launches increasing year over year.

To support research in this CubeSat domain, the \abbreviationFull[Air Force Institute of Technology]{AFIT} has a space vehicle design series of courses that guides students through the Systems Engineering process using a satellite system. Starting with a set of mission objectives, the design teams perform trade studies, generate requirements, design the CubeSat system, and perform verification and validation of those requirements with physical components over the span of three courses. This process mirrors the real-world development process, but on a much faster timeline. 

As design teams begin the development process of a CubeSat, there can be a steep learning curve. Many engineers are not familiar with \abbreviationFull[Model-Based Systems Engineering]{MBSE} tools or methodologies, and teams need to start their designs from scratch. Reference Architectures exist in other domains to capture best practices and provide a starting point for new systems, so this thesis attempts to develop and demonstrate a Reference Architecture for the CubeSat domain. By providing CubeSat designers with a template, including automatically generating tables and documentation, they can focus more on the design and less on learning how to use and organize the complicated model. Additionally, by providing a component library to use and pre-built analysis tools using those components, they can build off previous successful designs and rapidly simulate candidate solutions. Thorough documentation and guidance included in the Reference Architecture will also increase standardization amongst the team. 

