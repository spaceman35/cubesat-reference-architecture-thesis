There are many examples of Reference Architectures used in the commercial sector, but this section will focus on Reference Architectures that were developed at AFIT or are otherwise relevant to the CubeSat domain. 

First, the \abbreviationFull[Small Unmanned Aircraft System]{SUAS} Reference Architecture developed at AFIT will be investigated. This is a relevant example as it fulfills the same general goals as the CubeSat Reference Architecture; namely, that it is for use in a design course series and is intended for students to use as a template for their design efforts. This SUAS Reference Architecture was started before this CubeSat effort and provides a useful baseline and inspiration, even if it is for a different domain. AFIT professors Dr. Jacques and Dr. Cox developed this architecture using Cameo Systems Modeler to describe a generic SUAS as described by the Army Research, Development, and Engineering Command, focused primarily on specific product output for the SUAS specialization track \citep{Jacques2019}. The SUAS Reference Architecture contains a Basic Ground Station Model, a Basic Multi-Rotor System Model, a Component Library, and sample build using the architecture. The Component Library function was a unique function that will be adapted to this CubeSat Reference Architecture. The SUAS Reference Architecture is designed to allow students to easily build to a design specification from COTS components in the Component Library and test those designs using built-in parametric diagrams. These concepts will be applied to the CubeSat Reference Architecture as well, adapted for use in the spacecraft design course series. 

\textcolor{red}{Put a figure from their RA here too, maybe showing their top level hierarchy package showing the subsections or a figure showing the component library?}

Jacques and Cox focused on the SUAS culture of rapid prototyping, and the Reference Architecture allows for designs to be developed at a much faster pace. The common template and vision provided through the model helps interdisciplinary teams design, build, and test SUAS systems with more time spent on producing a quality product, and less time spent designing the entire model from scratch \citep{Jacques2019}. Jacques and Cox captured their own extensive SUAS experience into their Reference Architecture, and the model will continue to be improved over time. Currently, it is being improved to streamline the cumbersome DoD Cybersecurity Risk Assessment process. The component library will also continually evolve as COTS components change. The component library, parametric diagrams, and general organization are useful in the development of the CubeSat Reference Architecture, but the spacecraft design course series has some unique differences that will be tailored to. 

\textcolor{red}{I'm going to totally redo this analysis of the CRM so just ignore it for now.}

In the CubeSat domain, Kaslow, Ayres, et al (cite) built a CubeSat Reference Model (CRM) as part of a partnership between the \abbreviationFull[Object Management Group]{OMG} and the \abbreviationFull[International Council on Systems Engineers]{INCOSE} to help CubeSat developers by providing logical, reusable architecture elements at a high level. Some sample diagrams are provided in their interim status updates (cite them), but the actual Cameo model was not available to investigate. This CRM describes three levels of architectural foundation that are necessary to capture the whole domain: the enterprise level, the space and ground segments, and the space and ground subsystems. This is similar to the enterprise-organization-system structure of Army RDECOM, but has been adapted to be space domain specific.  Figure \ref{fig:CRM Domain} indicates the structure for the CubeSat domain as described by Kaslow et al.

\begin{figure}[!h]
    \centering
    \includegraphics[width=\textwidth]{Thesis/Literature_Review/Lit Review Figures/CubeSat Domain.png}
    \caption{CRM CubeSat Domain}
    \label{fig:CRM Domain}
\end{figure}

Kaslow et al. used a block definition diagram to demonstrate the hierarchy of elements within the domain. They depict the CubeSat Mission Enterprise as being directly composed of a Space Segment; a Ground Segment; Ground Station Services; and Transport, Launch, and Deploy Services. Furthermore, they are able to identify what must be developed by the CubeSat Project in greater detail, as shown by Figure \ref{fig:CRM RA Scope}.


\begin{figure}[!h]
    \centering
    \includegraphics[width=\textwidth]{Thesis/Literature_Review/Lit Review Figures/CubeSat RA scope.png}
    \caption{CRM Scope}
    \label{fig:CRM RA Scope}
\end{figure}


\begin{figure}[!h]
    \centering
    \includegraphics[width=\textwidth]{Thesis/Literature_Review/Lit Review Figures/CubeSat Ground Segment.png}
    \caption{CRM Ground Segment}
    \label{fig:CRM Ground Segment}
\end{figure}

\begin{figure}[!h]
    \centering
    \includegraphics[width=\textwidth]{Thesis/Literature_Review/Lit Review Figures/CubeSat RA Space Segment.png}
    \caption{CRM Space Segment}
    \label{fig:CRM Space Segment}
\end{figure}


Much like in Figure \ref{fig:CRM Domain}, Kaslow et al. have described all of the parts that a CubeSat is composed of and provided a Block Definition Diagram distinguishing between the Mission Payload and the Spacecraft Bus. This is done to show the necessary structure/components of a CubeSat (the Spacecraft Bus) and the on-orbit mission structure/components (the Mission Payload). This same process was also continued to formulate the composition for the Ground System Segment as well, using similar organization shown in Fig \ref{fig:CRM Ground Segment} as an example.

Kaslow et al. determined that this logical architecture would provide guidance for CubeSat developers to begin to formulate their own mission specific architectures, knowing that their model did not have and could not have the specificity required to support every type of mission. It provided a top-level guide to how a CubeSat enterprise is organized, and some of the external stakeholders as well, as shown in Fig \ref{fig:CRM Stakeholders}. Their model is a starting point for mission specific teams to incorporate their unique knowledge to formulate their own reference architectures.

After investigating the CRM status updates, however, the CubeSat Reference Model was missing much of the low-level details that was included in the SUAS Reference Architecture. A thorough reference architecture in this domain ought to include the high-level documentation and views of the CRM and the low-level component library and functionality of the SUAS reference architecture. 


\begin{figure}[!h]
    \centering
    \includegraphics[width=\textwidth]{Thesis/Literature_Review/Lit Review Figures/CubeSat Stakeholders.png}
    \caption{CRM Stakeholders}
    \label{fig:CRM Stakeholders}
\end{figure}


Several other gaps exist that will be addressed in this thesis effort. First, the CRM is not designed for outputting traditional documents for system level reviews. There is no easy way to generate a \abbreviationFull[Concept of Operations]{CONOPS} document or \abbreviationFull[Operational Requirements Document]{ORD}, for example, and that is a desire for an AFIT CubeSat Reference Architecture. Second, the CRM does not appear to have a component library or a generic, logical system that can be easily adapted by students new to MBSE. Finally, the CRM does not appear to have sufficiently detailed value properties for the system to be useful for detailed mission analysis using MATLAB and STK. Students in the AFIT course series must design down to a very specific level of detail with many value properties for each subsystem in order to perform the required analysis and calculations.


\textcolor{red}{Now discuss the Farrell attempt or the Firefly model from Friedenthal? I think it'd be more beneficial to discuss the Friedenthal model as that was the primary inspiration for my model so far.}