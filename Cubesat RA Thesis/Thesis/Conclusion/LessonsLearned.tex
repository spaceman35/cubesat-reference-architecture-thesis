%Lessons Learned

Over the course of this research, there were several lessons learned that warrant discussion. First, developing a Reference Architecture should not be a solitary endeavor. The early phases of this project were done primarily alone, but the most progress was made when other opinions were taken into consideration. Additionally, the model should be geared towards the key stakeholders, not just the modeler's preferences. This was made apparent during demonstrations to faculty members, whose opinions are the most important for this effort. Some design choices made sense originally, but needed modifications after seeing the greater context of the course objectives. 

Another lesson learned was to embrace the cloud environment for collaboration. By using the cloud environment, multiple people could be making edits at the same time, and changes are reflected for all users once they are committed. Its well worth the effort in setting up the cloud environment, getting each team member an account, and walking through the best practices for cloud modeling at the very start of the project. 

Finally, throughout the design process, the issues caused by copying and pasting blocks within a model became apparent. If a user copies and pastes an entire model (the Generic CubeSat Model for instance), everything seems to work perfectly. However, if a user copies just one internal package over to a different model, issues start popping up where you least expect them. Making a Reference Architecture that includes multiple full models within requires careful consideration before copying elements from one model to another. 
