Designing a spacecraft is a daunting and complex endeavor. Due to the nature of space launch, most spacecraft only get one chance at success, and spacecraft can take many years and millions of dollars to develop. As such, modeling, simulation, and testing are vital for a space vehicle program's success, and finding new ways to mature technologies and flight test them can improve this process. The CubeSat-class of nanosatellite can help with this problem, by providing a cost-effective platform to mature technologies or even perform operational missions as part of a CubeSat constellation. This thesis attempts to assist design teams rapidly develop and prototype these CubeSat designs. 

Dr. Will Roper, assistant secretary of the Air Force for acquisition, has emphasized the need for a faster acquisition cycle and for bolder ideas. During the Air Force Association’s Air, Space and Cyber Conference in 2019, Dr. Roper said “To become a more competitive acquisition system, the Air Force needs to be aware of trends in technology. The world is changing. We have to change with it. The key is to decide which technology will be successful and being able to act on those trends with a system that is leaner, meaner and faster than our opponents.” \citep{Roper2019} In the space domain, CubeSats are that latest technological "leaner and meaner" trend, and the US Air Force and Space Force are embracing it.

To support research in this CubeSat domain, the \abbreviationFull[Air Force Institute of Technology]{AFIT} has a space vehicle design series of courses that guides students through the Systems Engineering process using a satellite system. The course sequence guides students throughout the space vehicle engineering process. Starting with a set of mission objectives, the design teams perform trade studies, generate requirements, design the CubeSat system, and perform verification and validation of those requirements with physical components over the course of the three courses. This process mirrors the real-world development process, but on a much faster timeline. 

In an effort to improve this rapid-prototyping environment, this thesis demonstrates the usage of a new Reference Architecture to guide CubeSat design teams through the whole design process, hopefully speeding up the process and improving the quality of designs in the end. The first usage of this Reference Architecture will be the AFIT space vehicle design series, but the Reference Architecture should be useful to any CubeSat design team to start from. 

