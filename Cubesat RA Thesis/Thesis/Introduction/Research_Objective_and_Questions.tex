In an effort to improve this rapid-prototyping environment, this thesis demonstrates the usage of a new Reference Architecture to guide CubeSat design teams through the whole design process, hopefully speeding up the process and improving the quality of designs in the end. The first usage of this Reference Architecture will be the AFIT space vehicle design series, but the Reference Architecture should be useful to any CubeSat design team as a starting template. \\

\noindent \textbf{The research objectives are:}

\begin{enumerate}
\item{Create a practical and useful Reference Architecture for rapidly-prototyping CubeSat designs.}
\item{Create easy-to-use document generators that use model elements to generate traditional system level review documentation.}
\item{Present this Reference Architecture to AFIT instructors for feedback.}
\item{Lay the groundwork for future analysis work with STK and MATLAB integration for more comprehensive mission analysis using model elements.}
\end{enumerate}

\noindent \textbf{The research questions are:}

\begin{enumerate}
\item{What are the tools necessary to perform mission modeling using model-based systems engineering?}
\item{What viewpoints are most useful to common stakeholders?}
\item{How can useable documentation be generated from only model elements, keeping the source of truth within the model?}
\item{What needs to be done in the model to allow for external tools (STK, MATLAB, etc.) to interact with the MBSE tool?}
\item{Can cloud-based collaboration improve the MBSE design process for interdisciplinary teams?}

\end{enumerate}