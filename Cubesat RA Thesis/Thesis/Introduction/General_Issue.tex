
%Provides background/context/motivation for the research and need to research the problem
%Provides sufficient support (hints at) identifying the potential solution you are proposing in the research hypothesis

%note the difference between \cite and \citep (for if it's Last Name (year) or (last name,year), respectively)

Designing a spacecraft is a daunting and complex endeavor. Due to the nature of space launch, most spacecraft only get one chance at success, and spacecraft can take many years and millions of dollars to develop. As such, modeling, simulation, and testing are vital for a space vehicle program's success, and finding new ways to mature technologies and flight test them can speed up this process. The CubeSat-class of nanosatellite can help with this problem, by providing a cost-effective and speedy platform to mature technologies or even to perform operational missions as part of a CubeSat constellation. 

Dr. Will Roper, assistant secretary of the Air Force for acquisition, has been emphasizing the need for a faster acquisition cycle and for bolder ideas. During the Air Force Association’s Air, Space and Cyber Conference in 2019, Dr. Roper said “To become a more competitive acquisition system, the Air Force needs to be aware of trends in technology. The world is changing. We have to change with it. The key is to decide which technology will be successful and being able to act on those trends with a system that is leaner, meaner and faster than our opponents.” \citep{Roper2019} In the space domain, CubeSats are that latest technological "leaner and meaner" trend, and the US Space Force is embracing it. 

To support research in this CubeSat domain, the \abbreviationFull[Air Force Institute of Technology]{AFIT} has a Space Vehicle Design series of courses that guides students through the Systems Engineering process using a satellite system. The first course, ASYS 531, gives graduate students a list of mission objectives, requirements, and constraints for a theoretical CubeSat mission. They use those to create system requirements including orbit design, external interface diagrams, and preliminary technical budgets. The next course, ASYS 631, identifies the CubeSat’s subsystem requirements and concludes with a design (e.g. hardware selection and computer aided drawings) of a proposed solution. Manufacturing and testing of prototype hardware and software to validate those previous requirements is the emphasis of the third course, ASYS 632.

With the assistance of modeling tools, those student groups could evaluate and simulate their proposed designs as part of the verification process, as well as easily output traditional system-level documents for external stakeholders and major review activities. In the end, this Reference Architecture should support this spacecraft design course series and contribute to higher quality system designs. The Reference Architecture should also be useful to other academic and government organizations involved in CubeSat development, as the architecture is general enough for use outside of AFIT. 

