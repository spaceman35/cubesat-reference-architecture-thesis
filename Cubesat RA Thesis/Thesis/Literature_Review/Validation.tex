When preparing to develop the CubeSat Reference Architecture, a Digital Engineering (DE) Validation Tool from SAIC was made available. This free tool was developed by SAIC and is provided free to the public as a set of validation rules and customizations for Cameo. SAIC states "Our free system model validation tool guides modeling consistency to reduce errors, aid analyses, and improve quality." This appeared to be a useful addition to this CubeSat Reference Architecture, so it was closely examined. 

SysML provides a vast array of modeling options and styles, and this DE tool aims to limit the language and standardize modeling techniques. By using this tool, a team can ensure that each team member is using the same diagram types, the same flow structures, the same definitions, etc., all of which align with the goals of a Reference Architecture. 

However, the strict limits on SysML diagrams and modeling techniques did not match how AFIT students learn or have practiced in their preceding courses. In a commercial company with a specific modeling culture, this DE toolset would be more useful to get engineers on the same page. Their engineers may have come from different modeling backgrounds and it is important to establish a common modeling style, but for new modelers who just learned the basics of SysML, this tool was too restrictive and unnecessary in this author's view. Students using this model also all learned SysML from the same institution and already have a relatively common modeling vocabulary and level of expertise. 

For example, the validation will output hundreds of errors on basic models that are missing the fine levels of details expected by this rule set. This rule set requires very precise port structures and object flows throughout, and this Reference Architecture just isn't meant to exactly simulate the complex behavior of the CubeSat. Other tools are being developed, like the Integrated Mission Modeling Tool, that incorporate MATLAB's Simulink Stateflow and STK simulations to accomplish those goals. The state machine diagrams and activity diagrams in this Reference Architecture just need to convey what is happening. Furthermore, the rules as provided don't allow for standard practices that AFIT students have learned to use, such as Swim Lanes, including attachments, and using various types of ports. This rule set does not really allow for any blocks to have empty fields such as "Documentation" or "Rationale," but this Reference Architecture does not necessarily require those fields everywhere. Furthermore, the sheer quantity of errors is frustrating for a new modeler who isn't familiar with the nuances of SysML yet. 

Some rules were useful to replicate just for double checking work though, such as ensuring requirements have names and text, checking that activity diagrams have a starting and ending point, ensuring fork and merge nodes are complete, and ensuring every element is named. These rules were imported into the validation profile of the Reference Architecture, but it is a small portion of the restrictions in the original tool.

While the DE toolset was not fully used in this edition of the CubeSat Reference Architecture, the tool could be much more useful if AFIT develops its own style guide and best practices. If there is an agreement on SysML usage as an institution, the rule set can be modified and improved to incorporate these practices. The rules as presented were just too restrictive for the primary audience using this CubeSat Reference Architecture. Furthermore, the rules are geared towards a much more refined and polished model than the ones created by students in a short timeframe. 